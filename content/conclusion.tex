\chapter{总结与展望}
\section{工作总结}
本文研究了基于深度强化学习的智能交通信号调度问题,包括单路口场景以及多路口场景下的信号调度问题。首先我们总结了一些传统的交通信号控制方法,这些方法要么受制于一些严苛的假设条件导致难以应用于实际交通控制中,要么缺乏对实时交通状况的考虑导致无法高效地缓解交通拥堵的情况。
然后我们总结了一些已有的使用深度强化学习解决智能交通信号调度问题的工作,在不同的调度场景下(单路口场景和多路口场景),我们对这些已有工作的进行了系统的分析,并指出他们存在的不足,最后我们针对这些不足提出新的解决方案。

对于单路口场景下的交通信号控制,已有的基于学习的方法更多的注重于提高通行效率,而忽略了公平性问题。在本文中,我们提出了一个新的模型可以在保证通行效率的同时,兼顾到对公平性的考虑。通过大量的实验验证了我们方法的有效性。

对于多路口场景下的交通信号控制,我们使用了IRL with communication的框架,并提出了一种新的将道路网建模成图的建模方式,在这种建模方式下,智能体在提取邻近节点的信息时可以剔除那些对自己无用的信息。通过实验验证,我们的方法在通行效率和收敛性上都优于已有的方法。

\section{未来展望}
虽然目前有很多使用强化学习来解决智能交通信号调度的工作,但是这些工作多数都只停留在仿真阶段,即实验场景的搭建以及效果的验证都是通过仿真器完成的,要想将这些模型部署到现实生活中的信号灯上还需要更多的研究和实地测试。但是随着车联网技术的发展使得我们可以实时地获取道路上车辆的信息,加上最近人工智能技术的崛起,
一些基于学习的交通信号调度算法可以在与环境的交互中不断的提升自身的性能,这些因素为进一步实现真实道路场景下的智能交通信号控制提供了技术支持。