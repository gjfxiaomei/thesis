\chapter{总结与展望}
本文研究了基于深度强化学习的智能交通信号控制,并对已有工作在不同场景下的不足进行了分析,并各自提出了新的方法。

对于单路口场景下的交通信号控制,已有的基于学习的方法更多的注重于提高通行效率,而忽略了公平性问题。在本文中,我们提出了一个新的模型可以在保证通行效率的同时,兼顾到对公平性的考虑。通过大量的实验验证了我们方法的有效性。

对于多路口场景下的交通信号控制,我们使用了IRL with communication的框架,并提出了一种新的将道路网建模成图的建模方式,在这种建模方式下,智能体在提取邻近节点的信息时可以剔除那些对自己无用的信息。通过实验验证,我们的方法在通行效率和收敛性上都优于已有的方法。