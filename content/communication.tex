
\chapter{多路口场景下交通信号调度}
对于多路口的交通信号调度问题,协作(Coordination)可以有效地提升整体通行效率,以下列出几种常见的协作策略:
\begin{table}[htb]
    \caption[协作策略]{常见的协作策略\label{tab:coordination}}
    \begin{tabular}{clp{0.4\columnwidth}}
      \toprule
      协作策略 & 目标 & 说明 \\
      \midrule
      Global single agent & $max_{\mathbf{a}}Q(s, \mathbf{a})$ & $s$是全局的环境状态,$\mathbf{a}$是所有路口的联合动作。\\
      Independent RL without Communication & $max_{a_{i}}\sum{i}Q_{i}(o_i, a_i)$ & $o_i$是路口$i$的局部观测,$a_i$是路口$i$的动作。\\
      Independent RL with Communication & $max_{a_i}\sum{i}Q_i(\Omega(o_i, \mathcal{N}_i), a_i)$ &$\mathcal{N}_i$是路口$i$的邻近路口的状态表示,$\Omega(o_i, \mathcal{N}_i)$是整合路口$i$及其邻近路口状态表示的函数。\\
      \bottomrule
    \end{tabular}
\end{table}

\section{相关工作}

\section{已有工作中的不足}

\section{改进}

