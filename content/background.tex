% 本文件是示例论文的一部分
% 论文的主文件位于上级目录的 `bachelor.tex` 或 `master.tex`

\chapter{研究背景}
% \section{研究背景}
随着汽车制造业的快速发展以及城市化进程的推进,我国汽车保有量在不断的增加,交通拥堵情况也在不断恶化,极大的影响了人们的生活的城市的运作,同时这种拥堵现象也在向中小城市蔓延。为了缓解交通拥堵,很多城市也提出了不同的解决方法,有减少出行车辆数量的“限号”政策,也有通过加快城市道路建设来加大城市交通承载量的方法。其实,交通拥堵通常是由于不同的车流为了争夺同一个“行驶资源”而造成的。这一“行驶资源”通常就是车辆所处的交通路口,所以现代城市交通管理的装主要主要方法是在道路汇合的交叉路口安装信号灯并通过简单的策略来调度通过的车流,已到达减少交通拥塞的目的。但是随着车辆数量的不断增加,之前传统的交通信号控制策略已经难以应对现在更加复杂的交通模式。因此,如何制定出更加高效和智能的交通信号调度策略显得格外的重要。

随着车联网技术的发展,对于实时车辆数据的获取变得越来越容易,利用得到的车辆数据可以获得实时的交通状况,并且如何根据实时的交通状况来制定最优的策略一直是研究的热点。以往多数的研究是采用基于优化的方法,根据车流的状况计算出一个最优的信号灯的相位序列,但是这种方法要求车流的状况是比较简单的,例如服从均匀分布,与现实中的车流情况相比太过理想化,所以难以部署到实际场景中。伴随着人工智能技术的发展,一些研究者提出利用深度强化学习来控制信号灯,将整个交通信号灯控制建模成一个马尔可夫决策过程(Markov Decision Process)。对于每一次决策,输入当前的交通状况作为状态,输出一个作用在信号灯上的动作,例如变换到下一个相位(phase)。这种方法对于车流的情况没有限制,通过在大量不同的仿真车流下进行训练可以得到一个鲁棒的模型,能够应对不同的车流场景并做出最优的决策,并且这种方法在通行效率上也比基于优化的方法和传统的规则控制方法更高。


\section{研究意义}
传统的交通控制方法在当下已经难以有效地减轻交通拥堵的情况,因为其更多的是通过一些预先设定的规则和一些根据历史数据总结出的经验来控制信号灯,没有考虑实时的交通状况,但是由于其简单以及易于部署的特点,绝大多数城市的信号灯都还在采用这种控制模式。

随着人工智能技术的飞速发展以及越来越多的城市交通数据,政府和企业正在积极寻求改善交通系统的智能交通信号控制解决方案。与传统交通信号控制不同的是,智能交通信号控制会根据实时的交通状况做出最优的决策并以此来控制信号灯的变化,已达到最大程度地减轻交通拥堵的目的。另一方面,随着最近强化学习技术的发展,我们看到学术界对使用强化学习来改善交通信号控制的热情越来越高涨,并且也提出了很多基于深度强化学习的智能交通信号控制方法,但是任然有很多问题需要研究。本文主要尝试解决已有工作中遗留的两个问题,即公平性问题和多路口场景下的协调通信问题。

公平性问题是指,不同车辆通过同一个路口所需的通行时间可能有很大的差别,因为信号灯可能为了提高整体通行效率而牺牲一些车辆,让这些车辆多等待一些时间,即便这些车辆可能是先进入路口的,这对这些车来说是不公平的。一个好的控制策略应该在提高通行效率的同时能够保证每辆车所需的通行时间大致相同,也就是说,车辆通行时间的方差应该越小越好。但是已有的工作都是使用车辆的平均通行时间来衡量通行效率,很自然的忽略了公平性问题。

在多路口交通信号控制问题中,路口之间通过通信来进行数据交互可以实现多个路口的协同控制,从而提高整个路网的通行效率。但是已有的工作在进行数据交互时,笼统地将所有的信息传递给目标路口,会导致目标路口难以提取出有效的信息,从而增加学习的难度,甚至学习出错误的策略。