\chapter{特色功能}

本章节介绍由 \nuaathesis{} 提供的特有的宏。

\section{定理环境}

\nuaathesis{} 没有定义任何定理环境,
但提供了三个宏 \cs{nuaatheorem(g|chap|chapu)} 来定义不同编号方法的定理环境。
\begin{enumerate}
  \item \cs{nuaatheoremg} 的编号只有一个数字;
  \item \cs{nuaatheoremchap} 的编号由“章节.序号”构成,不同的定理环境的编号是独立的,
  它们的数字编号会重复,如“\autoref{ex:oneplus}”后面可能出现“\autoref{non:dora}”;
  \item \cs{nuaatheoremchapu} 的编号也是由“章节.序号”构成,
  但它们的数字编号是统一的,同一个数字不会重复出现(仅限用\cs{nuaatheoremchapu}声明的定理环境之间)。
  如“\autoref{def:distance}”后面\textbf{不会}出现“假设~2.1”,但可能出现“定义~2.2”或“\autoref{assume:fail}”;
\end{enumerate}

由于学校没有规定计数的编号,所以所有的定理环境应该由作者来决定编号方式,
这也意味着所有的定理环境都要由作者来定义(这不是 \nuaathesis{} 在偷懒哦)。

顺便一提,在同一章里同时出现两种编号方式的定理环境,很可能造成混乱,
所以请合理安排定理环境的编号方式。以下开始举栗子。

\subsection*{样例}

\begin{definition}[欧几里得距离]
\label{def:distance}
点$\mathbf{p}$与点$\mathbf{q}$的\textbf{欧几里得距离},是连接该两点的线段($\overline{\mathbf{pq}}$)的长度。

在笛卡尔坐标系下,如果 $n$维欧几里得空间下的两个点 $\mathbf{p}=(p_1, p_2, \dots, p_n)$ 与点
$\mathbf{q} = (q_1, q_2, q_3, \dots, q_n)$,那么点$\mathbf{p}$与点$\mathbf{q}$的距离,
或者点$\mathbf{q}$与点$\mathbf{p}$的距离,由以下公式定sds义:
\begin{align}
\label{equ:1}
d(\mathbf{p},\mathbf{q}) = d(\mathbf{q},\mathbf{p}) & = \sqrt{(q_1-p_1)^2 + (q_2-p_2)^2 + \cdots + (q_n-p_n)^2} \\
\label{equ:2}
& = \sqrt{\sum_{i=1}^n (q_i-p_i)}
\end{align}
\end{definition}

\begin{proof}
由\cs{nuaatheorem(g|chap|chapu)}定义的定理环境支持 \cs{autoref},
比如在\autoref{def:distance}中,\autoref{equ:2}是\autoref{equ:1}的简写。

但是 \cs{autoref} 只能在 \cs{ref} 加上前缀,无法加上后缀。
所以上一句话的后半部分,更推荐手工来写标注 “(\ref{equ:2}) 是 (\ref{equ:1}) 的简写”。

定理环境里面可以换行,不过证明与其他定理环境稍有不同,它是单独定义实现的,
因此末尾会有一个(帅气的) QED 符号。
\end{proof}

\begin{assumption}
\label{assume:fail}
假设本身就不成立
\end{assumption}

\begin{lines}
\label{s1}
例句1
\end{lines}

\section{参考文献}
\label{sec:bib}
参考文献应该以上标的形式标注于论述之后,就像这样:

\begin{itemize}
\item 研究表明\cite{r1},早睡早起有益身体健康。
\item 如果想同时引用多个文献\cite{r2,r3,r4,r6},只需要在 \verb|cite{}| 中用逗号分开\texttt{citeKey}就好。
\end{itemize}

本模板保留了 \cquthesis{} 里的 \texttt{inlinecite},但请注意它不符合学校的要求,无论本科还是硕士、博士,
请\textbf{谨慎}使用:
文献\inlinecite{r6}表明,文献\inlinecite{r7,r8,r9}所述的情况是有理论依据的。

\nuaathesis 格式测试,学校的参考文献格式并不是 GB7714-2015,所以追加一些测试样例。
《要求》里列出的格式有:
\begin{enumerate}
  \item 连续出版物\cite{n11,n12}:[序号]作者.文题.刊名,年,卷号(期号):起~止页码.
  \item 专译集\cite{n21,n22}:[序号]作者.书名(译者).出版地:出版者,出版年:起~止页码.
  \item 论文集\cite{n31,n32}:[序号]作者.文题.编者,文集名,出版地:出版者,出版年:起~止页码.
  \item 学位论文\cite{n41,n42,n43}:[序号]姓名.文题,[XX学位论文].授予单位所在地:授予单位,授予年.
  \item 专利\cite{n51,n52,n53}:[序号]申请者.专利名,国名,专利文献种类,专利号,出版日期.
  \item 技术标准\cite{n61,n62,n63}:[序号]发布单位,技术标准代号,技术标准名称,出版地:出版者,出版日期.
\end{enumerate}

注:目前实现的格式仍然与《要求》有点差异:
\begin{enumerate}
  \item 《要求》里论文集的编者、文集名、出版地是逗号分隔,而目前是点号分隔;
  \item 《要求》的学位论文用中文注明学位,目前没实现;
  \item 在信息缺失的情况下,《要求》貌似直接把对应字段省略,目前仍显示“XX不详”。
\end{enumerate}
